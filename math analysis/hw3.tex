\documentclass[a4paper,11pt]{article}

\usepackage{amsmath}
\usepackage{amssymb}
\usepackage{cmap}
\usepackage{geometry}
\usepackage{hyperref}
\usepackage{indentfirst}
\usepackage{xeCJK}
\usepackage{titlesec}

\geometry{margin=1in}

\setCJKmainfont{Adobe Song Std}
\setCJKfamilyfont{hei}{Adobe Heiti Std}
\setmainfont{Times New Roman}
\newcommand{\hei}{\CJKfamily{hei}}

\title{第三次数分作业}
\author{罗翔宇}
\date{\today}

\begin{document}

\maketitle

\section*{习题2}
\subsection*{(1)}
错误

比如数列$x_n=\frac{1}{10^n}$, 取$a=\frac{1}{100^n}$, 则当$n>100$时都有$|\frac{1}{10^n}-a|<\frac{1}{10^{100}}$, 但是$a$不是数列的极限
\subsection*{(2)}
正确
\subsection*{(3)}
正确
\section*{习题3}
\subsection*{(1)}
$\forall \varepsilon>0$, 取$N=[\frac{1}{\varepsilon}]+1$, 则
\begin{align*}
	|\frac{\cos n}{n}| &\le |\frac{\cos(\frac{1}{\varepsilon} +1)}{\frac{1}{\varepsilon}+1}|\\
	&\le |\frac{\varepsilon}{\varepsilon+1}\cos(\frac{1}{\varepsilon}+1)|\\
	&\le \varepsilon
\end{align*}
所以$\displaystyle\lim_{n\to \infty}\frac{\cos n}{n}=0$
\subsection*{(5)}
令$z_n=\frac{1}{n}, y_n=n^4q^n, t_n=\frac{y_{n+1}}{y_n}$

则$t_n=q\frac{(n+1)^4}{n^4}=q+q(\frac{4}{n}+\frac{6}{n^2}+\frac{4}{n^3}+\frac{1}{n^4})$

显然$t_n$递减,且$\displaystyle \lim_{n\rightarrow \infty}t_n=0$, 那么必存在一个$k > 0$, 使得$\forall n > k, t_n < 1$, 即当$n > k$时, $y_i$单调递减

又因为$\forall n \in \mathbb{N}, y_n > 0$, 所以$y_n$是无穷小量, 又因为$z_n$是无穷小量, 所以$y_nz_n$为$O(1)$

即$\displaystyle \lim_{n\rightarrow \infty}x_n=0$
\section*{习题7}
因为$\forall \varepsilon > 0$, $\exists N_0$, 使得$\forall n>N_0$, $|x_n-a|<\varepsilon$

令$N=[\frac{N_0}{2}]+1$, 所以$\forall \varepsilon > 0, \forall n>N, 2n > N_0, 2n+1 > N_0, |x_{2n}-a|<\varepsilon, |x_{2n+1}-a|<\varepsilon$

必要性得证

$\forall \varepsilon > 0, \exists N_1 \in \mathbb{N}$使得$\forall n>N_1, |x_{2n}-a|<\varepsilon$

$\forall \varepsilon > 0, \exists N_2 \in \mathbb{N}$使得$\forall n>N_2, |x_{2n+1}-a|<\varepsilon$

取$N=\max \{2N_1,2N_2+1\}$, $\forall \varepsilon > 0, \forall n>N$都有$|x_n-a|<\varepsilon$

充分性得证
\section*{习题10}
\subsection*{(1)}
$\displaystyle \lim_{n\rightarrow \infty} \cos n \sin \frac{a}{n} = 0$, 证明如下:

令$y_n=\cos n, z_n=\sin \frac{a}{n}$

显然$y_n$是有界数列, 又因为$\displaystyle \lim_{n\rightarrow \infty}|\frac{a}{n}|=0$, 所以$\displaystyle\lim_{n\rightarrow \infty}z_n=0$, 即$z_n$是无穷小量

所以$x_n=y_nz_n$, $x_n$是无穷小量, $\displaystyle \lim_{n\rightarrow \infty} \cos n \sin \frac{a}{n} = 0$得证
\subsection*{(3)}
\begin{align*}
	x_n&=\frac{1}{n^2}+\frac{2}{n^2}+\dots+\frac{n}{n^2}\\
	&=\frac{n(n+1)}{2n^2}\\
	&=\frac{1}{2}+\frac{1}{2n}
\end{align*}

显然$\frac{1}{2n}$为无穷小量,所以$\displaystyle \lim_{n\rightarrow \infty}x_n=\frac{1}{2}$
\subsection*{(5)}
\begin{align*}
	\displaystyle x_n&=\frac{\sqrt[3]{n^2}\sin n^2}{n+1}\\
	&=\frac{\sin n^2}{\sqrt[3]{n}+\frac{1}{n^{\frac{2}{3}}}}\\
	&\le \frac{\sin n^2}{\sqrt[3]{n}}
\end{align*}

显然$\sin n^2$是有界数列, 又因为$\displaystyle \lim_{n\rightarrow \infty} \sqrt[3]{n} = \infty$

所以$\displaystyle \lim_{n\rightarrow \infty}\frac{\sin n^2}{\sqrt[3]{n}} = 0$

又因为$x_n > 0$, 所以$\displaystyle \lim_{n\rightarrow \infty}x_n=0$
\subsection*{(6)}
\begin{align*}
	x_n&=\frac{1}{1\cdot2}+\frac{1}{2\cdot3}+\cdots+\frac{1}{n(n+1)}\\
	&=(1-\frac{1}{2})+(\frac{1}{2}-\frac{1}{3})+\cdots+(\frac{1}{n}-\frac{1}{n+1})\\
	&=1-\frac{1}{n+1}
\end{align*}
显然$\displaystyle \lim_{n\rightarrow \infty}\frac{1}{n+1}=0$, 所以$\displaystyle \lim_{n\rightarrow \infty}x_n=1$
\subsection*{(9)}
$\displaystyle \lim_{n\rightarrow \infty}\sqrt[n]{a}=1$, 证明如下:

$\forall \varepsilon > 0$, 只需取$N = [\frac{1}{\log_a{1-\varepsilon}}]+1$

即可得到$\forall n > N, \sqrt[n]{a} > 1-\varepsilon$, 即$|\sqrt[n]{a} - 1| < \varepsilon$

所以原命题得证
\end{document}
