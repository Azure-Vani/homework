\documentclass[a4paper,11pt]{article}

\usepackage{amsmath}
\usepackage{amssymb}
\usepackage{cmap}
\usepackage{geometry}
\usepackage{hyperref}
\usepackage{indentfirst}
\usepackage{xeCJK}
\usepackage{titlesec}

\geometry{margin=1in}

\setCJKmainfont{Adobe Song Std}
\setCJKfamilyfont{hei}{Adobe Heiti Std}
\setmainfont{Times New Roman}
\newcommand{\hei}{\CJKfamily{hei}}

\title{第二次数分作业}
\author{罗翔宇}
\date{\today}

\begin{document}

\maketitle

\section*{习题15}
由$\inf f(x) \inf g(x) \le f(x)g(x)$, 可得$\inf f(x) \inf g(x) \le \inf [f(x)g(x)]$

因为$\forall x_0 \in D, f(x_0)g(x_0) \le \sup f(x) g(x_0)$

所以$\sup f(x) \inf g(x) \ge \inf[f(x)g(x)]$

即$\inf f(x) \inf g(x) \le \inf[f(x)g(x)] \le \sup f(x) \inf g(x)$
\section*{习题18}
\subsection*{(1)}
$f(x)=\sqrt{|x|}$
\subsection*{(2)}
$f(x)=
\begin{cases}
	\sqrt{x},&x \ge 0\\
	\sqrt{-x},&x < 0
\end{cases}$
\subsection*{(3)}
$f(x)=\sqrt{x - [x]}$
\section*{习题25}
假如$\sin (x^2+x)$为周期函数,且最小正周期为$t$

可得 $$\sin(x^2+x) = \sin((x+t)^2+x+t)$$

分别取$x=-1$和$x=0$代入, 即可得$\sin (t^2-t)=0$, 和$\sin (t^2+t)=0$

所以
\begin{equation}
t^2+t=p\pi
\end{equation}
\begin{equation}
t^2-t=q\pi, p, q\in\mathbb{Z}
\end{equation}

联立(1)(2)可解得$\pi=\frac{2(p+q)}{(p-q)^2}$, 与$\pi$是无理数矛盾

故假设不成立, 即$\sin (x^2+x)$为非周期函数
\section*{习题26}
\subsection*{(1)}
不失一般性, 不妨设$x_2 \ge x_1$, 令$g(x)=\frac{f(x)}{x}$, 显然$g(x)$为单调递减函数

则

$$f(x_1+x_2) \le f(x_1)+f(x_2)$$
$$\Leftrightarrow (x_1+x_2)g(x_1+x_2) \le x_1g(x_1)+x_2g(x_2)$$
\begin{equation}
\Leftrightarrow x_1g(x_1+x_2)+x_2g(x_1+x_2) \le x_1g(x_1)+x_2g(x_2)
\end{equation}

因为$x_1, x_2 > 0$, 所以$g(x_1+x_2) \le g(x_1), g(x_1+x_2) \le g(x_2)$

所以不等式$(1)$成立, 原不等式得证
\subsection*{(2)}
不失一般性, 不妨设$x_2 \ge x_1$, 令$g(x)=\frac{f(x)}{x}$, 显然$g(x)$为单调递增函数

则

$$f(x_1+x_2) \ge f(x_1)+f(x_2)$$
$$\Leftrightarrow (x_1+x_2)g(x_1+x_2) \ge x_1g(x_1)+x_2g(x_2)$$
\begin{equation}
\Leftrightarrow x_1g(x_1+x_2)+x_2g(x_1+x_2) \ge x_1g(x_1)+x_2g(x_2)
\end{equation}

因为$x_1, x_2 > 0$, 所以$g(x_1+x_2) \ge g(x_1), g(x_1+x_2) \ge g(x_2)$

所以不等式$(1)$成立, 原不等式得证
\section*{习题29}
当$x=0$, $\frac{x}{1+x^2}=0$

当$x>0$, $\frac{x}{1+x^2}=\frac{1}{x+\frac{1}{x}} > 0$

令$g(x)=x+\frac{1}{x}$, 由$2\sqrt{ab} \le a+b$可得$\forall x \in (0, \infty), g(x)$最小值为 2, 即$f(x), \forall x \in (0, \infty)$最大值为$\frac12$

当$x<0$, 同理可得$f(x), \forall x \in (-\infty, 0)$最小值为$-\frac12$

综上, $f(x), \forall x \in (-\infty, \infty)$最大值为$\frac12$, 最小值为$-\frac12$

故$f(x)$在$(-\infty, \infty)$上有界
\end{document}
