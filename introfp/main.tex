\documentclass[xcolor=dvipsnames, 11pt]{beamer}

\usepackage{geometry}
\usepackage{xeCJK}
\usepackage{listings} 


\geometry{left=1cm}
%Set fonts
\setCJKmainfont{WenQuanYi Micro Hei Mono}
\setCJKfamilyfont{hei}{Adobe Heiti Std}
\setCJKmonofont{DejaVu Sans Mono}
\setmainfont{Times New Roman}

\setlength{\baselineskip}{0.5em}
\newcommand{\hei}{\CJKfamily{hei}}
\newenvironment{codetext}{\begin{semiverbatim}\begin{footnotesize}}{\end{footnotesize}\end{semiverbatim}}
\newenvironment{code}{\begin{block}{Code}\begin{semiverbatim} \begin{footnotesize}}{\end{footnotesize}\end{semiverbatim}\end{block}}
\newenvironment{iit}{\begin{itemize}\setlength{\itemsep}{0.2cm}}{\end{itemize}}

\usetheme{Frankfurt}
\usepackage{beamerthemesplit}
\setbeamertemplate{navigation symbols}{}
\setbeamertemplate{itemize items}[triangle]
\setbeamertemplate{enumerate items}[default]
\useoutertheme{infolines}
\setbeamertemplate{footline}[slide number]

\title{函数式编程与Haskell}
\subtitle{Introduction to functional programming}
\author{Vani}
\institute{Peking University}
\date{\today}

\begin{document}
\section{函数式编程}

\begin{frame}
	\titlepage
\end{frame}

\subsection{什么是函数式编程}

\begin{frame}[fragile]{什么是函数式编程?}
	你一定知道面向对象
	\begin{itemize}
	\vspace{0.1cm}
		\item<2-> 封装
	\vspace{0.1cm}
		\item<3-> 继承
	\vspace{0.1cm}
		\item<4-> 多态
	\vspace{0.1cm}
		\item<5-> 总之好厉害
	\end{itemize}
	\vspace{0.3cm}
	\uncover<6->{函数式}
	\vspace{0.1cm}
	\begin{itemize}
		\item<7-> 这是什么东西???
	\end{itemize}
\end{frame}

\begin{frame}[fragile]{其实...}
	函数式编程是一种完全不同的编程形式,与之对应的是指令式编程。

	\vspace{0.1cm}
	特点:
	\vspace{0.1cm}
	\begin{itemize}
		\item<2-> 不可变量
			\vspace{0.1cm}
		\item<3-> 惰性求值
			\vspace{0.1cm}
		\item<4-> 高阶函数
			\vspace{0.1cm}
		\item<5-> 无副作用
			\vspace{0.1cm}
		\item<6-> 一切皆函数
	\end{itemize}
\end{frame}

\subsection{lambda 演算}

\begin{frame}[fragile]{从停机问题开始}
	什么是停机问题?
	\vspace{0.2cm}
	\begin{itemize}
		\item<2->[]{\color{blue}{给定任意一个程序及起输入,判断该程序是否能在有限的计算以内结束}}
	\end{itemize}
	\vspace{0.5cm}
	\uncover<3->{作用?}
	\vspace{0.2cm}
	\begin{iit}
	\item<4->[]轻松愉悦地证明:
		\item<5->哥德巴赫猜想
		\item<6->费马大定理
		\item<7->......
	\end{iit}
\end{frame}

\begin{frame}[fragile]{如果真有这个算法}
	\uncover<2->{假如真的作出了这个算法,你只要给任意一个函数和这个函数的输入,它就能告诉你这个函数能不能结束}

	\vspace{0.6cm}
	\uncover<3->{我们用下面代码描述:}
	\vspace{0.2cm}
\begin{itemize}\item<4->[]\begin{code}
		bool halting(func, input) \{
		    return if_func_will_halt_on_input;
		\}
\end{code}\end{itemize}
\end{frame}

\begin{frame}[fragile]{充分利用停机判定}
	\uncover<2->{我们构造另一个函数:}
	\vspace{0.2cm}
\begin{iit} \item<3->[]\begin{code}
void evil(func) \{
    if (halting(func, func)) \{
        for(;;);
    \}
\}
	  \end{code}
\end{iit}
\vspace{0.2cm}
\uncover<3->{接下来,调用:\color{blue}{evil(evil)}}
\end{frame}

\begin{frame}[fragile]{悖论}
	\uncover<2->{到底停不停机??}

	\vspace{0.3cm}
	\uncover<3->{所以停机问题不可判定}
\end{frame}

\begin{frame}[fragile]{$\lambda$ 演算法}
基本语法:
\vspace{0.2cm}
\begin{iit}
	\item<2->$<expr>\;::=\;<identifier>$
	\item<3->$<expr>\;::=\;\lambda\ <identifier-list>.\;<expr>$
	\item<4->$<expr>\;::=\;(<expr>\ <expr>)$
\end{iit}
\vspace{0.4cm}
\uncover<5->{比如:}
\begin{iit}
	\item<6->[]$\lambda\ x\ y.\;x\;+\;y$
\end{iit}
\end{frame}

\begin{frame}[fragile]{$\lambda$ 演算公理}
\uncover<2->{置换公理}
\begin{iit}
\item<3->$\lambda\ x\ y.\ x\;+\;y\ =>\ \lambda\ a\ b.\ a\;+\;b$
\end{iit}
\vspace{0.5cm}
\uncover<4->{代入公理}
\begin{iit}
\item<5->$(\lambda\ x\ y.\ x\;+\;y)\ a\ b\ =>\ a\;+\;b$
\end{iit}
\vspace{0.5cm}
\uncover<6->{以上就是$\lambda$ 演算的全部公理系统}
\end{frame}

\begin{frame}[fragile]{函数生成器}
\uncover<2->{我们可以使用$\lambda$ 演算法定义各种各样的语法}
\begin{iit}
	\item<3->[]\begin{block}{"and" operator:}\begin{semiverbatim} \begin{footnotesize}
		\uncover<4->{let And = 
		        \\ True False. False
		        \\ True True. True
		        \\ False True. False
		        \\ False False. False}
		\end{footnotesize}\end{semiverbatim}\end{block}
	\item<5->[]\begin{block}{"if" statement:}\begin{semiverbatim} \begin{footnotesize}
		\uncover<6->{let if = \\cond tv fv. (cond and tv) or (not cond and fv)}
		\end{footnotesize}\end{semiverbatim}\end{block}
\end{iit}
\end{frame}

\subsection{Y组合子}

\begin{frame}[fragile]{似乎还有缺陷}
\uncover<2->{如何实现递归?}

\vspace{0.2cm}
\uncover<3->{写一个计算阶乘的函数...}
\begin{iit}
	\item<4->[]\begin{alertblock}{Code}\begin{codetext}
	let fac = \\n. if n == 0 then 1 else n * fac(n-1)
	\end{codetext}\end{alertblock}
\end{iit}
\vspace{0.4cm}
\uncover<5->{\color{red}Error! "fac" has not been defined}
\end{frame}

\begin{frame}[fragile]{把自身参数化}
\uncover<2->{为了实现自己调用自己,不妨传一个参数进入}
\vspace{0.2cm}
\begin{iit}
\item<3->[]\begin{code}
let fac = \\self n. if n == 0 then 1 else n * self(self,n-1)
\end{code}
\vspace{0.3cm}
\uncover{只不过传了一个函数重复调用而已}
\end{iit}
\end{frame}

\begin{frame}[fragile]{柯里化}
	\uncover<2->{思考, 如果一个两个参数的函数,我们只传给它一个参数,会得到什么?}
	\vspace{0.2cm}
	\begin{description}
		\item<3->[柯里化]是把接受多个参数的函数变换成接受一个单一参数(最初函数的第一个参数)的函数,并且返回接受余下的参数而且返回结果的新函数的技术
	\end{description}
	\vspace{0.2cm}
	\uncover<4->{有没有想到C++里的函数适配器bind1st, bind2nd}
\end{frame}

\begin{frame}[fragile]{不动点}
\begin{iit}
\item<2->{假如我们已经得到了一个完美的阶乘函数FAC}
\item<3->{考虑一个有缺陷的阶乘函数:}
\begin{iit}\item<3->[]\begin{code}
let fac = \\self n. if n == 0 then 1 else n * self(n-1)
\end{code}\end{iit}
\item<4->{如果把FAC传给fac会发生什么?}
\item<5->{\color{red}{FAC = fac(FAC)}}
\item<6->{FAC是有缺陷的fac的不动点!}
\end{iit}
\end{frame}

\begin{frame}[fragile]{构造真正的递归函数}
\begin{iit}
\item<2->如果我们有一个神奇的函数Y,可以得到所有伪递归函数的真正递归函数
\item<3->即:Y(F) = f
\item<4->Y(F) = f = F(f) = F(Y(F))
\item<5->如何构造这样的Y呢?
\end{iit}
\end{frame}

\begin{frame}[fragile]{Y组合子}
\begin{iit}
\item<2->考虑下面的一个函数:
\item<3->[]\begin{code}let FAC_gen = \\self. fac(self(self))
\end{code}
\item<4->嵌套一下?
\item<5->[]\begin{code}FAC_gen(FAC_gen) = fac(FAC_gen(FAC_gen))
\end{code}
\item<6->不妨令FAC = FAC\_gen(FAC\_gen)
\item<7->fac(FAC) = FAC!
\end{iit}
\end{frame}

\begin{frame}[fragile]{Y组合子}
\begin{iit}
\item<2->[]于是对于每个伪递归,都找出一个类似FAC\_gen这样的函数就行了
\item<3->[]\begin{exampleblock}{Y combinator}\begin{codetext}
	let Y = \\f. let gen = \\self. f(self(self)); return gen(gen)
\end{codetext}\end{exampleblock}
\item<4->[]展开看一下:
\end{iit}
\end{frame}

\begin{frame}[fragile]{验证}
\begin{iit}
\item<2->首先定义有缺陷版本:
\item<3->$let\ fac\ =\ \backslash\ self\ n.\ if\ n\ ==\ 0\ then\ 1\ else\ n\ *\ self(n\ -\ 1)$
\item<4->$Y(fac)\ =>$
\item<5->[]$let\ gen\ =\ \backslash\ self.\ \textbf{fac}(self(self));\ return\ gen(gen)$
\item<6->$Y(fac)\ =>\ gen(gen)\ =>\ fac(gen(gen))$
\item<7->$fac(gen(gen))\ (n)\ =>\ if\ n\ ==\ 0\ then\ 1\ else\ n\ *\ \textbf{gen(gen)}(n-1)$
\item<8->$gen(gen)\ =>\ fac(gen(gen))$
\item<9->$\textbf{fac(gen(gen))}(n)\;=>\;$
\item<10->[]$if\;n\;==\;0\;then\;1\;else\;n\;*\;\textbf{fac(gen(gen))}(n-1)$
\end{iit}
\end{frame}

\begin{frame}[fragile]{图灵等价}
\begin{iit}
\item<2->Y组合子的推导成功相当于在$\lambda$ 演算公理中添加了一条公理
\vspace{0.4cm}
\begin{iit}
\item<3->[]\large{可以在定义函数的过程中引用自身}
\end{iit}
\vspace{0.4cm}
\item<4->于是可以证明,$\lambda$ 演算系统和图灵机是等价的
\item<5->那和图灵停机问题等价的问题是什么呢?
\vspace{0.4cm}
\begin{iit}
\item<6->[]\large{不存在一个算法能够判定任意两个$\lambda$ 函数是否等价}
\end{iit}
\vspace{0.4cm}
\item<7->证明?
\end{iit}
\end{frame}

\section{haskell编程}
\subsection{初识haskell}

\begin{frame}[fragile]{初识haskell}
\begin{iit}
\item<2->特性:
\begin{iit}
\item<3->部分求值(柯里化)
\item<4->惰性求值
\item<5->无副作用
\item<6->....
\end{iit}
\end{iit}
\end{frame}

\subsection{基本语法}
\begin{frame}[fragile]{第一个函数}
\begin{iit}
\item<2->计算阶乘:
\item<3->[]\begin{code}
fac :: Num a => a -> a
fac 0 = 1
fac n = n * fac(n - 1)
\end{code}
\vspace{0.2cm}
\item<4->模式匹配
\item<5->类型限制
\end{iit}
\end{frame}

\begin{frame}[fragile]{列表}
\begin{iit}
\item<2->类似python里的列表
\item<3->":"操作符和模式匹配可以很方便地处理列表
\item<4->让我们自己来实现一个map函数
\item<4->[]\begin{code}
map' _ [] = []
map' f (x:xs) = f x:map f xs
\end{code}
\item<5->试验一下
\item<6->[]\begin{code}
Prelude> map' (+3) [1,2,3]
[4,5,6]
\end{code}
\end{iit}
\end{frame}

\begin{frame}[fragile]{列表产生器}
\begin{iit}
\item<2->列表产生器是什么?
\item<3->[]\begin{code}
Prelude> [x | x<-[1..10], x `mod` 2 == 0]
[2,4,6,8,10]
\end{code}
\item<4->学过数学的都能看明白吧
\item<5->自己来实现一个快速排序:
\item<6->[]\begin{code}
qsort [] = []
qsort (x:xs) = qsort[y | y<-xs, y <= x] ++ [x] ++ qsort[y | y <-xs, y > x]
Prelude> qsort [1,3,5,3,2]
[1,2,3,3,5]
\end{code}
\end{iit}
\end{frame}

\begin{frame}[fragile]{函数的高级用法}
\begin{iit}
\item<2->使用.来复合两个函数:$f.g x = f(g(x))$
\item<3->使用\$来改变函数求值顺序
\item<4->综合运用:
\item<4->[]\begin{code}
main = interact \$ concatMap(++"\\n").takeWhile(/="42").lines
\end{code}
\item<5->[]\begin{code}
main=interact\$(\\[x,y]->zipWith(\\a b->if a==b then '0' else '1')x y).lines
\end{code}
\end{iit}
\end{frame}

\subsection{输入与输出}
\begin{frame}[fragile]{输入和输出}
\begin{iit}
\item<2->haskell里,用于执行输入和输出的函数被定义为IO action类型
\item<3->[]\begin{code}
main = do 
    putStrLn "what's your name?" 
    name <- getLine 
    putStrLn ("Your name is " ++ name)
\end{code}
\item<4->putStrLn和getLine都为IO string类型
\end{iit}
\end{frame}

\begin{frame}[fragile]{输入和输出}
\begin{iit}
\item<2->getLine其实类似与一个盒子,我们要使用<-从里面提取一个string
\item<3->使用=号是达不到预期效果的
\item<4->[]\begin{code}
main = do 
    name <- getLine
    name' = getLine
\end{code}
\item<5->name'其实是一个函数,相当于getLine
\item<6->于是可以执行name<-name'来从name'里提取字符串了
\end{iit}
\end{frame}

\begin{frame}[fragile]{输入和输出}
\begin{iit}
\item<2->我们把一个IO action绑定到main函数,并在程序开始执行时触发
\item<3->并使用do语句把若干个IO action函数绑定为一个
\item<4->[]\begin{code}
main = do 
    line <- getLine 
    if null line 
        then return () 
        else do 
            putStrLn \$ revverseWords line 
            main 
\end{code}
\item<5->注意return不是终止程序执行,而是返回一个IO action
\end{iit}
\end{frame}

\subsection{type和typeclass}
\begin{frame}[fragile]{Type和typeclass}
\begin{iit}
\item<2->Type即为通常意义下的类型,比如Int,Bool等
\item<3->一个typeclass定义了一些函数,所有该typeclass的实例都支持这些函数
\item<4->比如Eq的实例都支持判断是否相等,Ord的实例都支持比较大小
\end{iit}
\end{frame}

\begin{frame}[fragile]{Type和typeclass}
\begin{iit}
\item<2->我们先试着定义一个自己的Type
\item<3->[]\begin{code}
data Tree a = Empty | Node a (Tree a) (Tree a) deriving(Show, Read, Eq)
\end{code}
\item<3->此时Tree是什么类型呢?
\item<4->Tree a才是一个合法的类型
\item<5->Tree不过是一个类型构造子
\end{iit}
\end{frame}

\begin{frame}[fragile]{kind:类型的类型}
\begin{iit}
\item<2->kind即为类型的类型,可用:k命令来检测某类型或者构造子的kind
\item<3->Bool的结果为*
\item<4->但是Tree的kind为*->* (有没有想到函数?)
\item<5->BTW,Tree类型的一个插入函数:
\item<6->[]\begin{code}
ins x Empty = Node x Empty Empty
ins x (Node a l r) 
    | x <= a = Node a (ins x l) r
    | otherwise = Node a l (ins x r)
\end{code}
\item<7-> 注意它是返回了一个新的Tree
\end{iit}
\end{frame}

\section{函数式编程的抽象}
\subsection{Functor}
\begin{frame}[fragile]{抽象:Functor typeclass}
\begin{iit}
\item<2->Functor typeclass的实例是可以被map的对象
\item<3->定义:
\item<4->[]\begin{code}
class Functor f where   
    fmap :: (a -> b) -> f a -> f b
\end{code}
\item<5->注意f并不是一个类型,而是一个类型构造子。其kind为*->*
\item<6->fmap接受一个函数,它把一个类型映射成另外一个
\end{iit}
\end{frame}

\begin{frame}[fragile]{Functor的例子}
\begin{iit}
\item<2->来看一下list是如何被定义为Functor的实例的
\item<3->[]\begin{code}
instance Functor [] where
    fmap = map
\end{code}
\vspace{0.2cm}
\item<4->另一个Functor的例子是I/O action:
\item<5->[]\begin{code}
instance Functor IO where 
    fmap f action = do 
        result <- action 
        return (f result)
\end{code}
\end{iit}
\end{frame}

\begin{frame}[fragile]{Functor的例子}
\begin{iit}
\item<2->IO就是一个类型构造子。
\item<3->:t getLine的结果为IO string
\item<4->:k IO的结果为*->*
\item<5->[]\begin{code}
main = do line<-fmap reverse getLine
          putStr line
\end{code}
\end{iit}
\end{frame}

\begin{frame}[fragile]{Functor的例子}
\begin{iit}
\item<2->(->)r 也是Functor的例子
\item<3->(->)r 究竟代表什么?
\item<4->a->r => (->) r a
\item<5->(->) r a是一个类型,a也是一个类型
\item<6->接受一个类型,返回一个新类型?
\item<7->所以(->)r是一个类型构造子
\item<8->于是(->)r也可以是Functor的实例
\end{iit}
\end{frame}

\begin{frame}[fragile]{Functor的例子}
\begin{iit}
\item<2->[]\begin{code}
instance Functor ((->) r) where   
    fmap f g = (\\x -> f (g x))
\end{code}
\item<3->我们已经知道fmap形态为fmap :: (a -> b) -> f a -> f b
\item<4->把所有的f在心里替换为(->) r
\item<5->于是就变成了fmap :: (a -> b) -> ((->) r a) -> ((->) r b)
\item<6->换成中缀函数的形式:fmap :: (a->b)->(r->a)->(r->b)
\item<7->这正是函数的复合!
\end{iit}
\end{frame}

\begin{frame}[fragile]{Functor law}
\begin{iit}
\item<2->Functor law:
\begin{iit}
\item<3->\Large{fmap的结果不改变原有Functor的拓扑顺序}
\end{iit}
\item<4->思考:如何把我们的Tree实例化为Functor
\item<5->[]\begin{code}
instance Functor Tree where
    fmap f Empty = Empty
    fmap f (Node a l r) = Node (f a) (fmap f l) (fmap f r)

*Main> fmap (*2) (ins 3 (Node 2 Empty Empty))
Node 4 Empty (Node 6 Empty Empty)
\end{code}
\end{iit}
\end{frame}

\subsection{Applicative functors}
\begin{frame}[fragile]{进一步抽象:Applicative functors}
\begin{iit}
\item<2->如果我们对一个Functor实例A fmap一个多參函数会发生什么?
\item<3->得到一个函数集合,函数之间保留A的拓扑顺序,不妨称这个函数集合类型为B
\item<4->怎么把B上的每个函数都对应地作用于A上?
\end{iit}
\end{frame}

\begin{frame}[fragile]{Applicative functor的定义}
\begin{iit}
\item<2->先围观Applicative functor的定义:
\item<3->[]\begin{code}
class (Functor f) => Applicative f where   
    pure :: a -> f a   
    (<*>) :: f (a -> b) -> f a -> f b
\end{code}
\item<4->f被型别限定为Functor类型
\item<5->pure接受一个值,返回一个包含那个值的Applicative functor
\item<6-><*>接受一个装有函数的Functor和另一个functor,然后取出第一个functor里的函数对第二个做map
\end{iit}
\end{frame}

\begin{frame}[fragile]{Applicative functor例子}
\begin{iit}
\item<2->考虑如何把Maybe类型作为Applicative实例的
\item<3->[]\begin{code}
instance Applicative Maybe where   
    pure = Just   
    Nothing <*> _ = Nothing   
    (Just f) <*> something = fmap f something
\end{code}
\item<4->pure的定义使用了柯里化噢
\item<5->如果左边是Just,那么<*>会抽出其中的函数来map右面的值
\item<6->如果任何一个函数是Nothing,那结果就为Nothing
\end{iit}
\end{frame}

\begin{frame}[fragile]{Applicative functor例子}
\begin{iit}
\item<2->试一下效果吧:
\item<3->[]\begin{code}
\uncover<3->{Prelude> Just (+3) <*> Just 9   
Just 12}
\uncover<4->{Prelude> Just (++"hahah") <*> Nothing   
Nothing}
\uncover<5->{Prelude> ghci> (++) <\$> Just "johntra" <*> Just "volta"   
Just "johntravolta"}
\end{code}
\item<6-><\$>为一个语法糖,类似于中缀版的fmap
\item<7->[]\begin{code}
(<\$>) :: (Functor f) => (a -> b) -> f a -> f b   
f <\$> x = fmap f x
\end{code}
\end{iit}
\end{frame}

\subsection{Monad}
\begin{frame}[fragile]{从Functor, Applicative functor到Monad}
\begin{iit}
\item<2->因为许多的形态都都可以被map,所以抽象出了Functor这个typeclass
\item<3->[]\begin{code}
fmap :: (Functor f) => (a -> b) -> f a -> f b
\end{code}
\item<4->然后只要针对Functor撰写实例就行了
\item<5->比如Maybe a, [a], IO a, 甚至是(->)r a
\end{iit}
\end{frame}

\begin{frame}[fragile]{从Functor, Applicative functor到Monad}
\begin{iit}
\item<2->接下来我们发现如果a->b也被包含在一个Functor里呢
\item<3->比如Just (*3)如何将它应用到Just 5上从而得到Just 15
\item<4->于是我们得到了Applicative functor:
\item<5->[]\begin{code}
(<*>) :: (Applicative f) => f (a -> b) -> f a -> f b
\end{code}
\end{iit}
\end{frame}

\begin{frame}[fragile]{从Functor, Applicative functor到Monad}
\begin{iit}
\item<2->现在问题来了:
\begin{iit}
\item<3->如何将一个具有context的值m a,丢进一个只接受普通值a的函数中
\end{iit}
\item<4->换句话说如何套用一个形态为a -> m b的函数至m a
\item<5->我们要求的函数为:
\item<6->[]\begin{code}
(>>=) :: (Monad m) => m a -> (a -> m b) -> m b
\end{code}
\end{iit}
\end{frame}

\begin{frame}[fragile]{Monad}
\begin{iit}
\item<2->我们先来看一下Monad的实例
\item<3->[]\begin{code}
class Monad m where   
    return :: a -> m a   
 
    (>>=) :: m a -> (a -> m b) -> m b   
 
    (>>) :: m a -> m b -> m b   
    x >> y = x >>= \\_ -> y   
 
    fail :: String -> m a   
    fail msg = error msg
\end{code}
\end{iit}
\end{frame}

\begin{frame}[fragile]{Monad}
\begin{iit}
\item<2->return是包装一个monadic value
\item<3->$(>>=)$ 运算符是把monadic value传给一个接受普通值的函数f,f的返回值是一个monadic value
\item<4->fail用于处理错误的情况
\item<5->list,Maybe,IO action都是一个Monad
\end{iit}
\end{frame}

\begin{frame}[fragile]{Maybe Monad}
\begin{iit}
\item<2->Maybe的例子:
\item<3->[]\begin{code}
Prelude> Just 9 >>= (\\x->return(x+3))
Just 12
Prelude> Just 9 >>= (\\x->return(x+3)) >>= (\\x->Nothing)
Nothing
\end{code}
\item<4->可以很好的处理Nothing这种情况
\end{iit}
\end{frame}

\begin{frame}[fragile]{list Monad}
\begin{iit}
\item<2->list其实也是一个Monad
\item<3->[]\begin{code}
instance Monad [] where   
    return x = [x]   
    xs >>= f = concat (map f xs)   
    fail _ = []
\end{code}
\item<4->例子:
\item<5->[]\begin{code}
Prelude> [1,2] >>= \\n -> ['a','b'] >>= \\ch -> return (n,ch)   
[(1,'a'),(1,'b'),(2,'a'),(2,'b')]
\end{code}
\item<6->我们从monadic value取出普通值给函数,$>>=$ 会帮我们处理好一切关于context的问题
\end{iit}
\end{frame}

\begin{frame}[fragile]{do语句}
\begin{iit}
\item<2->可以使用do语句把一些Monad的操作绑定在一起
\item<3->[]\begin{code}
f = do
    x <- Just 4
    y <- Just 5
    return (x, y)
\end{code}
\item<4->有没有想到IO操作
\end{iit}
\end{frame}

\begin{frame}[fragile]{函数的副作用}
\begin{iit}
\item<2->haskell里monad最重要的作用就是引入函数的副作用
\item<3->比如Maybe类型的Nothing情况
\item<4->比如IO action的输入输出操作
\item<5->比如Error类型的出错信息
\end{iit}
\end{frame}

\begin{frame}[fragile]{Monad law}
\begin{iit}
\item<2->三条Monad定律:
\begin{iit}
\item<3->$(return x) >>= f == f x$
\item<4->$m >>= return == m$
\item<5->$(m >>= f) >>= g == m >>= (\backslash x -> f x >>= g)$
\end{iit}
\item<6->第一二条很显然
\end{iit}
\end{frame}

\begin{frame}[fragile]{Monad law}
\begin{iit}
\item<2->第三条很类似于函数的复合
\item<3->让我们回顾一下函数的复合
\item<4->[]\begin{code}
(.) :: (b -> c) -> (a -> b) -> (a -> c)   
f . g = (\\x -> f (g x))
\end{code}
\item<5->于是f.(g.h)和(f.g).h显然等价的
\item<6->仿照函数的复合我们可以合成monadic function的复合规则:
\item<7->[]\begin{code}
(<=<) :: (Monad m) => (b -> m c) -> (a -> m b) -> (a -> m c)   
f <=< g = (\\x -> g x >>= f)
\end{code}
\item<8->到此我们就可以像操纵普通值和普通函数一样操纵monadic value和monadic function
\end{iit}
\end{frame}

\begin{frame}[fragile]{End.}
\begin{iit}
\item<2->更加高级的东西:
\begin{iit}
\item<3->Arrows. 
\item<4->Stream Fusion
\item<5->Zipper
\item<6->.....
\end{iit}
\item<7->欢迎大家自行阅读相关资料
\item<8->推荐作业:使用haskell实现一个平衡树
\end{iit}
\end{frame}

\begin{frame}[fragile]{End.}
\begin{iit}
\item<2-> 欢迎提问!
\end{iit}
\end{frame}
\end{document}
