\documentclass[xcolor=dvipsnames, 9pt]{beamer}

\usepackage{xeCJK}
\usepackage{listings} 


%Set fonts
\setCJKmainfont{WenQuanYi Micro Hei Mono}
\setCJKfamilyfont{hei}{Adobe Heiti Std}
\setCJKmonofont{DejaVu Sans Mono}
\setmainfont{Times New Roman}

\setlength{\baselineskip}{0.5em}
\newcommand{\hei}{\CJKfamily{hei}}
\newenvironment{code}{\begin{semiverbatim} \begin{footnotesize}}{\end{footnotesize}\end{semiverbatim}}

\usetheme{Frankfurt}
\usepackage{beamerthemesplit}
\setbeamertemplate{navigation symbols}{}
\setbeamertemplate{itemize items}[triangle]
\setbeamertemplate{enumerate items}[default]
\useoutertheme{infolines}
\setbeamertemplate{footline}[slide number]

\title{Play with Python}
\subtitle{Introduction to programming with python}
\author{Vani}
\institute{Peking University}
\date{\today}



\begin{document}

\begin{frame}[plain]
\titlepage
\end{frame}

\section{Introduction}

\begin{frame}[fragile]{quiz}
	猜一下下面程序的作用
	\begin{itemize}
		\item<9-> \textbf<9>{\alert<9>{Python}}
		\item<2->[]
			\begin{block}{}
				\begin{code}
					print ''.join([['0','1'][i!=j] for i,j in zip(raw\_input(),raw\_input())])
				\end{code}
			\end{block}
		\item<8-> Ruby
		\item<3->[]
			\begin{block}{}
				\begin{code}
					$><<"%0#{gets=~/$/}b"%($_.to_i(2)^gets.to_i(2))
				\end{code}
			\end{block}
		\item<7-> Haskell
		\item<4->[]
			\begin{block}{}
				\begin{code}
					main=interact$(\\[x,y]->zipWith(\\a b->if a==b then '0' else '1')x y).lines
				\end{code}
			\end{block}
		\item<6-> C
		\item<5->[]
			\begin{block}{}
				\begin{code}
					\fontsize{4pt}{0.7em}\selectfont{
					#include <stdio.h>
					#include <string.h>
					int main(int argc, char *argv[]) \{
					    char s1[100], s2[100];
					    int i;
					    scanf("\%s\\n\%s", s1, s2);
					    for (i = 0; i < strlen(s1); i++) \{
					        if (s1[i] == s2[i])
						      putchar('1');
				              else
						      putchar('0');
					    \}
					\}}
				\end{code}
			\end{block}
	\end{itemize}
\end{frame}

\begin{frame}{Features}
	\begin{itemize}
		\item<1-> 优势:
			\begin{itemize}
				\item<2-> 解释性
					\vspace{0.1cm}
				\item<3-> 可移植性
					\vspace{0.1cm}
				\item<4-> 丰富的库
					\vspace{0.1cm}
				\item<5-> 甜甜的语法糖
			\end{itemize}
		\vspace{0.4cm}
		\item<6-> 劣势
			\begin{itemize}
				\item<7-> 执行速度慢
					\vspace{0.1cm}
				\item<8-> 奇怪的面向对象机制
					\vspace{0.1cm}
				\item<9-> 命名混乱
			\end{itemize}
	\end{itemize}
\end{frame}

\section{基本语法}
\subsection{基本数据类型}
\begin{frame}[fragile]{数}
	\begin{columns}
	\column{.45\textwidth}
	\begin{itemize}
				\item<2-> 整数
					\vspace{0.3cm}
				\item<3-> 长整数
					\vspace{0.3cm}
				\item<5-> 浮点数
					\vspace{0.3cm}
				\item<6-> 复数
	\end{itemize}
	\column{.45\textwidth}
	\begin{block}{Code}
		\begin{code}
			\uncover<4->{>>> print 2**200

1606938044258990275541962092341162602
522202993782792835301376}


		\uncover<7->{>>> print (-3+4j)*(1-2j)

		5+10j}
		\end{code}
	\end{block}
	\end{columns}
\end{frame}

\begin{frame}[fragile]{字符串}
	\uncover<2->{字符串是一种特殊的列表。字符串可以由单引号,双引号以及三引号来声明, 其中单引号和双引号是完全相同的,三引号用来声明多行字符串}
	\vspace{0.2cm}
	\begin{columns}
	\column{.45\textwidth}
	\begin{itemize}
				\item<3-> 可以直接使用加号来连接字符串,用乘号重复字符串
					\vspace{0.1cm}
				\item<7-> Python支持格式化字符串,类似C里面printf()的字符串格式化
	\end{itemize}
	\column{.45\textwidth}
	\begin{block}{Code}\begin{code}
		\uncover<4->{>>> a="a+"
		>>> b='b'}
		\uncover<5->{>>> print a+b, b*3}
		\uncover<6->{a+b bbb}

	\uncover<8->{>>> c=74}
	\uncover<9->{>>> print "\%d is a lucky number."\%(c)
		74 is a lucky number.
	}
	\end{code}\end{block}
	\end{columns}
\end{frame}

\subsection{运算符与表达式}
\begin{frame}[fragile]{运算符}
	Python的大部分运算符与C的功能一样,下面简单介绍一下一些与C不同的运算符.
	\begin{columns}
		\column{.45\textwidth}
		\begin{itemize}
			\item<2-> 乘方运算符 **
				\vspace{0.1cm}
			\item<4-> 整除运算符 //
				\vspace{0.1cm}
			\item<6-> 逻辑运算符 and, or, not. 短路计算在Python里也适用
				\vspace{0.1cm}
			\item<8-> Python没有自增自减运算符,但有复合赋值运算符
				\vspace{0.1cm}
			\item<10-> 利用短路计算模拟问号操作符. \uncover<11->{\alert<11->{有缺陷!}}
		\end{itemize}
		\column{.45\textwidth}
		\begin{block}{Code} \begin{code}
			\uncover<3->{>>> print 2**10
			1024}

			\uncover<5->{>>> print 1.5//0.2
			7.0}

			\uncover<7->{>>> print True and not False
			True}

			\uncover<9->{>>> a=42
				>>> a\%=10
				>>> print a
			2}

			\uncover<11->{>>> a = -10
				>>> print a < 0 and -a or a
				10
			}
		\end{code}\end{block}
	\end{columns}
\end{frame}

\begin{frame}[fragile]{缩进}
	\begin{columns}
		\column{.45\textwidth}
		\begin{itemize}
			\item<2-> Python里使用缩进来决定语句的分组, 就像C++里的大括号
				\vspace{0.1cm}
			\item<3-> 不要忘记语句尾有一个冒号
				\vspace{0.1cm}
			\item<5-> 错误的缩进会引发错误
				\vspace{0.1cm}
			\item<8-> \alert<7->{缩进可以使用一个制表或者四个空格,但请不要混合使用制表和空格缩进}
		\end{itemize}
		\column{.45\textwidth}
	\begin{block}{Code}
			\begin{code}
				\uncover<4->{>>> if a < 10\alert<3>{:}
				...    print "a<10"
				... else\alert<3>{:}
				...    if a >= 10 and a < 20\alert{:}
				...        print "a >= 10 and a < 20"
				...    else\alert{:}
				...        print "a >= 20"
				  }
			\uncover<6->{\alert<5->{>>> i=5
				>>>  print "Value is",i}
			}
			\uncover<7->{print 'value is'
					^
				 IndentationError: unexpected indent}
			\end{code}
		\end{block}
	\end{columns}
\end{frame}

\subsection{控制流程}
\begin{frame}[fragile]{判断语句}
	\begin{columns}
		\column{.45\textwidth}
			\begin{itemize}
				\item<2-> if的条件判断可以不加括号
					\vspace{0.1cm}
				\item<3-> if后面可以接elif
					\vspace{0.1cm}
				\item<5-> \alert{Python里没有switch语句,可以使用if...elif来代替}
					\vspace{0.1cm}
				\item<7-> if的三元表达式:$$X\ if\ C\ else\ Y$$
			\end{itemize}
		\column{.45\textwidth}
		\begin{block}{Code}\begin{code}
			\uncover<4->{a=10
			if a < 10:
			    print "less than 10"
			elif a == 10:
			    print "equal to 10"
			else:
		    print "greater than 10"}

		
			\uncover<8->{>>> a = 10
				>>> print "even" if a \% 2 == 0 else "odd"
			even}
		\end{code}\end{block}
	\end{columns}
\end{frame}

\begin{frame}[fragile]{循环语句}
	Python里的有while循环语句和for循环语句

	其中while循环和C用法基本相同,而for循环用于遍历列表或者迭代器.
	\begin{columns}
		\column{.45\textwidth}
		\begin{itemize}
			\item<2-> while语句后可接else从句,else后的内容在循环结束后或者break后执行
				\vspace{0.1cm}
			\item<5-> for语句的格式为$$for\ [item]\ in\ [volume]:$$其中volume必须为可遍历的类型,比如列表,迭代器,而item必须与volume内元素类型相符 
				\vspace{0.1cm}
			\item<6-> for语句后也可接else
		\end{itemize}
		\column{.45\textwidth}
		\begin{block}{Code}\begin{code}
			\uncover<3->{a=74
				while a \% 42 != 0:
				    a+=1
				else:
			    print "a is", a}

			\uncover<4->{Output:
			a is 84}


			\uncover<7->{for i in range(0, 11):
			    print i}

			\uncover<8->{Output:
				1
				2
				3}
		\end{code}\end{block}
	\end{columns}
\end{frame}

\subsection{函数}

\begin{frame}[fragile]{定义函数}
	函数通过关键字def定义,后接一个函数标识符名称,然后跟一对圆括号,其中包括一写变量名
	\begin{columns}
		\column{.45\textwidth}
		\begin{itemize}
			\item<2-> 函数的参数列表不用指定类型
				\vspace{0.1cm}
			\item<5-> 调用时必须要加双括号
				\vspace{0.1cm}
			\item<6-> 函数内变量均为局部变量,可使用global关键字声明全局变量
				\vspace{0.1cm}
			\item<7-> \alert<7->{不推荐!}
				\vspace{0.1cm}
			\item<8-> 当参数为一个序列时,如果在序列前加一个*,则自动将序列分割成参数表
		\end{itemize}
		\column{.45\textwidth}
		\begin{block}{Code}\begin{code}
			\uncover<3->{>>> def Double(a):
				...    return a * 2
				>>> print Double(3)
				>>> print Double("kill")
				}\uncover<4->{6
				killkill
			    }
				\uncover<9->{>>> def f(a, b, c):
					...    return a + b + c
					>>> print f(*[1, 2, 3])
					6
			    }
			\end{code}\end{block}
	\end{columns}
\end{frame}

\begin{frame}[fragile]{lambda函数}
	可通过lambda关键字声明匿名函数, 格式为:$$lambda\ [arg1[,\ arg2,\ ...]]:\ expression$$
	\begin{columns}
		\column{.45\textwidth}
		\begin{itemize}
			\item<2-> lambda表达式的返回值即为一个函数
				\vspace{0.1cm}
			\item<4-> 思考:如何使用lambda函数来定义阶乘函数fac(x)
				\vspace{0.1cm}
		\end{itemize}
		\column{.45\textwidth}
		\begin{block}{Code}\begin{code}
			\uncover<3->{
				print (lambda x: x + 1)(41)

				Output:
				42
			}
		\end{code}\end{block}
	\end{columns}
\end{frame}

\begin{frame}[fragile]{lambda函数}
	\begin{itemize}
		\item<2-> 一个可行的实现:
			\vspace{0.1cm}
		\item<3->[]\begin{exampleblock}{Code}\begin{code}
# Construct assitant function
YA =lambda x: lambda y: y(lambda : x(x)(y))
YB = YA(YA)

# Factor function
fact0 = lambda self: lambda n: (n==0) and 1 or n * self()(n-1)
fact = YB(fact0)

# >>> fact(6)
# 720
		\end{code}\end{exampleblock}
		\vspace{0.1cm}
		\item<4-> 可自行学习函数式编程相关知识(lambda演算,Y combanitor...)
	\end{itemize}
\end{frame}

\subsection{数据结构}
\begin{frame}[fragile]{列表}
	\uncover<2->{列表用于储存一个序列的项目,可用一对方括号来声明, 利用下标索引来访问元素}
	\begin{columns}
		\column{.45\textwidth}
		\begin{itemize}
			\item<3-> 列表中各个元素的类型可以不同
				\vspace{0.1cm}
			\item<5-> 可以创建多维列表
				\vspace{0.1cm}
			\item<7-> 列表生成器
		\end{itemize}
		\column{.45\textwidth}
		\begin{block}{Code}\begin{code}
			\uncover<4->{>>> print [3, "123"]
				[3, "123"]
			}

			\uncover<6->{>>> a = [[1,2,3],[4,5,6]]
				print a[0][1]
				5
			}

			\uncover<8->{>>> print [i for i in range(0, 3)]
				[0, 1, 2]
			}
			\uncover<10->{>>> print [i for i in range(0, 11) \\
					...if i \% 3 == 0]
					[0, 3, 6, 9]
				}
		\end{code}\end{block}
	\end{columns}
\end{frame}

\begin{frame}[fragile]{列表相关函数}
	\begin{columns}
		\column{.45\textwidth}
		\begin{description}
			\item<2->[[l:r]] 用于获得一个列表的子序列
				\vspace{0.1cm}
			\item<5->[list.append(obj)] 用于向list添加一个对象obj
				\vspace{0.1cm}
			\item<7->[len(list)] 获得列表长度
				\vspace{0.1cm}
			\item<9->[list.sort] 列表内元素排序
		\end{description}
		\column{.45\textwidth}
		\begin{block}{Code}\begin{code}
			\uncover<3->{>>> a = [3,2,1,4,5,6,7]}
			\uncover<4->{>>> print a[1:4]
				[3,2,1]
			}

			\uncover<6->{>>> a.append(8)
				>>> print a
				[3,2,1,4,5,6,7,8]
			}

			\uncover<8->{>>> print len(a)
				8
			}
			\uncover<10->{>>> a.sort()
				>>> print a
				[1,2,3,4,5,6,7,8]
			}
		\end{code}\end{block}
	\end{columns}
\end{frame}

\begin{frame}[fragile]{元组}
	\uncover<2->{元组与列表十分类似,不过元组是不可添加元素的.  元组通过圆括号中用逗号分割的项目定义}
	\begin{columns}
		\column{.45\textwidth}
		\begin{itemize}
			\item<3-> 元组的定义中的圆括号是可以省略的
				\vspace{0.1cm}
			\item<4-> 于是Python中变量赋值与交换两个变量可写成如右所示:
				\vspace{0.1cm}
			\item<6-> zip函数接受n个等长序列作为参数,返回一个序列,其中第i项为由n个序列第i项构成的一个元组
				\vspace{0.1cm}
			\item<7-> 可以同时遍历若干的序列
		\end{itemize}
		\column{.45\textwidth}
		\begin{block}{Code}\begin{code}
			\uncover<5->{>>> a,b = 1,2
				>>> print a, b
				1 2
				>>> a,b = b,a
				>>> print a, b
				2 1
			}

			\uncover<8->{>>> a, b = [0,1,2],[3,4,5]
				>>> for i, j in zip(a, b):
				...     print i. j
				0 3
				1 4
				2 5
			}
		\end{code}\end{block}
	\end{columns}
\end{frame}

\begin{frame}[fragile]{字典}
	\uncover<2->{字典类似于C++ STL里的map,用于建立关键字到值的映射, 声明格式如下:$$d = \{key1: value1, key2: value1\}$$}
	\begin{columns}
		\column{.45\textwidth}
		\begin{itemize}
			\item<3-> 字典的items返回一个元组的列表,其中每个元组都包含一对项目——关键字与对应的值
				\vspace{0.1cm}
			\item<4-> 使用items来遍历字典:
		\end{itemize}
		\column{.45\textwidth}
		\begin{block}{Code}\begin{code}
			\uncover<5->{>>> a = {"a": 1, "b": 2, "c": 3}
				>>> for i, j in a.items():
				...     print i, j
				a 1
				b 2
				c 3
			}
		\end{code}\end{block}
	\end{columns}
\end{frame}

\subsection{内置函数}
\begin{frame}[fragile]{字符串相关函数}
	\begin{columns}
		\column{.45\textwidth}
		\begin{description}
			\item<2->[s.join(list)] 用于连接字符串
				\vspace{0.2cm}
			\item<4->[s.split(se)] 用于分割字符串
				\vspace{0.2cm}
			\item<6->[s.strip()] 用于去除字符串开始和结尾的空白字符
		\end{description}
		\column{.45\textwidth}
		\begin{block}{Code}\begin{code}
			\uncover<3->{>>> a = ["Hello", "World", "!"]
				>>> print "_".join(a)
				Hello_World_!
			}
			
			\uncover<5->{>>> print "Hello_World_!".split("_")
				["Hello", "World", "!"]
			}

			\uncover<6->{>> print "  Hello ! ".strip()
				Hello !
			}
		\end{code}\end{block}
	\end{columns}
\end{frame}

\begin{frame}[fragile]{序列相关函数}
	\begin{columns}
		\column{.45\textwidth}
		\begin{description}
			\item<2->[map(func, list)] 将func作用list内每个元素,返回结果组成的序列
				\vspace{0.1cm}
			\item<4->[filter(func, list)] 将func作用于list内每个元素,返回令func为真的元素组成的序列
				\vspace{0.1cm}
			\item<6->[reduce(func, list)] 通过二元函数func将list缩减为一个值
				\vspace{0.1cm}
			\item<8->[sum(list), max(list), min(list)] 字面意思
		\end{description}
		\column{.45\textwidth}
		\begin{block}{Code}\begin{code}
			\uncover<3->{>>> a = [1, 2, 3]
				>>> print map(lambda x: x * 2, a)
				[2, 4, 5]
			}

			\uncover<5->{>>> a = [1, 2, 3, 4]
				>>> print filter(lambda x: x \%2 == 0, a)
				[2, 4]
			}

			\uncover<7->{>>> a = [1, 2, 3, 4]
				>>> print reduce(lambda x, y: x * y, a)
				24
			}
		\end{code}\end{block}
	\end{columns}
\end{frame}

\begin{frame}[fragile]{输入和输出}
	Python一般使用raw\_input和print来标准输入和输出
	\begin{columns}
		\column{.45\textwidth}
		\begin{itemize}
			\item<2-> raw\_input()默认读入一行字符串,需要使用int()函数来强制转化为整数
				\vspace{0.1cm}
			\item<4-> print默认在输出后面添加换行
				\vspace{0.1cm}
			\item<5-> 取消换行:
		\end{itemize}
		\column{.45\textwidth}
		\begin{block}{Code}\begin{code}
			\uncover<3->{>>> a = int(raw_input())
				>>> b = int(raw_input())
				>>> print a + b
				1
				3
				4
			}

			\uncover<6->{>>> print 3,
				>>> print 4
			3 4}
		\end{code}\end{block}
	\end{columns}
\end{frame}

\subsection{小技巧}
\begin{frame}[fragile]{Trick}
	\begin{itemize}
		\item<1-> 如何方便地读入一行内空格分割的整数
			\vspace{0.1cm}
		\item<2->[]\begin{exampleblock}{Code}\begin{code}
				>>> a,b,c = map(int, raw_input().split())
				1 2 3
				>>> print a + b + c
				6
			\end{code}\end{exampleblock}
	\end{itemize}
\end{frame}

\begin{frame}[fragile]{Trick}
	\begin{itemize}
		\item<1-> 读入两行长度相同的字符串,相同位输出1,不同位输出0
			\vspace{0.1cm}
		\item<2->[]\begin{exampleblock}{Code}\begin{code}
				>>> print "".join(["0","1"][i == j] for i, j in zip(raw_input(), raw_input()))
				abacb
				bbabb
				01101
			\end{code}\end{exampleblock}
	\end{itemize}
\end{frame}


\begin{frame}[fragile]{Trick}
	\begin{itemize}
		\item<1-> 转置一个矩阵
			\vspace{0.1cm}
		\item<2->[]\begin{exampleblock}{Code}\begin{code}
				>>> a = [[1,2,3],[4,5,6],[7,8,9]]
				>>> print zip(*a)
				[(1, 4, 7), (2, 5, 8), (3, 6, 9)]
			\end{code}\end{exampleblock}
	\end{itemize}
\end{frame}

\begin{frame}[fragile]{Trick}
	\begin{itemize}
		\item<1-> 快速排序
			\vspace{0.1cm}
		\item<2->[]\begin{exampleblock}{Code}\begin{code}
				>>> def qsort(a):
				...    return [] if a == [] else qsort(filter(lambda x: x <= a[0], a[1:])) + \\
					[a[0]] + qsort(filter(lambda x: x > a[0], a[1:]))

				>>> print qsort([3, 2, 4, 1, 5])
				[1, 2, 3, 4, 5]
			\end{code}\end{exampleblock}
	\end{itemize}
\end{frame}

\subsection{面向切面的编程}
\begin{frame}[fragile]{修饰器}
	\begin{columns}
		\column{.45\textwidth}
		\begin{itemize}
			\item<1-> 先来围观一个函数:
				\vspace{0.1cm}
			\item<3-> 怎样得到它的运行时间
				\vspace{0.1cm}
			\item<5-> \alert<5>{无法扩展 :(}
		\end{itemize}
		\column{.45\textwidth}
		\begin{exampleblock}{Code}\begin{code}
			\only<2-3>{def foo():
			    print "Hello World!"

			foo()
			}
			
			\only<4->{import time
					def foo():
					    start = time.clock()
					    print "Hello World!"
					    end = time.clock()
					    print "Time used:", end - start

					foo()
			}
		\end{code}\end{exampleblock}
	\end{columns}
\end{frame}

\begin{frame}[fragile]{修饰器}
	\begin{columns}
		\column{.45\textwidth}
		\begin{itemize}
			\item<1-> 如果可以有一个获取其他函数运行时间的函数...
				\vspace{0.1cm}
			\item<3-> 要修改函数调用的形式 
				\vspace{0.1cm}
			\item<4-> 直接获取一个新的函数!
				\vspace{0.1cm}
			\item<6-> Python的语法糖
				\vspace{0.1cm}
			\item<8-> 良好的扩展性
		\end{itemize}
		\column{.45\textwidth}
		\begin{exampleblock}{Code}\begin{code}
	\only<2-4>{import time
					def foo():
					    print "Hello World!"

					def getTime(func):
					    start = time.clock()
					    func()
					    end = time.clock()
					    print "Time used:", end - start

			\alert<3>{getTime(foo)}
		  }\only<5-8>{import time
					def getTime(func):
					    def wrapper():
					        start = time.clock()
					        func()
					        end = time.clock()
					        print "Time used:", end - start
					    return wrapper

					\only<5-6>{\alert<6>{def foo():
					    print "Hello World!"
					foo = getTime(foo)}}\only<7->{\alert<7>{@getTime
						def foo():
				    print "Hello World!" }}

					foo()
					\only<8->{
						@getTime
						def foo2():
						    print "Enjoy Python!"

					foo2()}}
		\only<9->{Output:

				Hello World!
				Time used: 0.0
				Enjoy Python!
			Time used: 0.0}
	    \end{code}\end{exampleblock}
	\end{columns}
\end{frame}

\begin{frame}[fragile]{带参数的修饰器}
	\begin{columns}
		\column{.45\textwidth}
			\begin{itemize}
					\item<2-> 修饰器可不可以传递参数?
						\vspace{0.1cm}
					\item<3-> 多了一层处理嵌套而已
						\vspace{0.1cm}
					\item<4-> 实质是面向切面的编程
			\end{itemize}
		\column{.45\textwidth}
		\begin{exampleblock}{Code}\begin{code}
			\only<3-4>{
				\alert<3>{def deco(arg):}
				    def getTime(foo):
				        def wrapper():
				            start = time.clock()
				            foo()
				            end = time.clock()
				            print arg, "used time:", end - start
				        return wrapper
				    \alert<3>{return getTime}

				@deco("foo1")
				def foo1():
				    print "Hello World!"

				@deco("foo2")
				def foo2():
				    print "Enjoy Python!"

				foo1()
				foo2()
			}
			\only<5->{Output:

				Hello World!
				foo1 used time: 0.0
				Enjoy Python!
				foo2 used time: 0.0
			}
		\end{code}\end{exampleblock}
	\end{columns}
\end{frame}

\subsection{库扩展}
\begin{frame}{常用的库}
	\begin{description}
		\item<2->[numpy] 科学计算包,包括一个强大的数组对象Array和线性袋鼠,傅立叶变换等常用函数, 一般与scipy库配套使用
			\vspace{0.1cm}
		\item<3->[scipy] 科学计算包,包括最优化,线性代数,积分,插值,图像处理,常微分方程等常用计算的模块
			\vspace{0.1cm}
		\item<4->[re] 正则表达式库,内置一个正则表达式引擎,可用于做一些复杂的字符串分析
			\vspace{0.1cm}
		\item<5->[urllib, urllib2]: 网络编程库,可方便地抓取网页,避免烦琐的socket编程
	\end{description}
\end{frame}

\subsection{更多}
\begin{frame}{更高端的Python}
	\begin{itemize}
		\item<2-> 面向对象的编程,封装继承相关
			\vspace{0.1cm}
		\item<3-> 闭包
			\vspace{0.1cm}
		\item<4-> 魔术方法
	\end{itemize}
	\uncover<5->{感兴趣的同学可自行查阅相关资料}
\end{frame}

\begin{frame}{End.}
	\huge{欢迎提问!}
\end{frame}

\end{document}

